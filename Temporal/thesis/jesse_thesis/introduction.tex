%%%%%%%%%%%%%%%%%%%%%%%%%%%%%%%%%%%%%%%%%%%%%%%%%%%%%%%%%%
\section{Introduction}
\label{sec:introduction}
%%%%%%%%%%%%%%%%%%%%%%%%%%%%%%%%%%%%%%%%%%%%%%%%%%%%%%%%%%

% 1. Problem my system is trying to solve:
Temporal expression resolution is the task of mapping from a temporal expression in text to a ground calendar date, set of dates, or duration. Temporal phrases are ubiquitous. They appear in all kinds of text, from e-mail to newswire to Wikipedia articles. 
Much of the information in print and Internet media consists of text that describes events occurring at a specific point in time. For a computer to be able to comprehend this text, it will need to be able to recognize events, and reason about how they are located temporally. Building a system that can understand temporal relations in text is a first step toward a system that can understand much of the information available in news articles, scientific journals and on the Internet.

Improved reasoning about times and events will enable us to build more effective systems for question answering, information extraction, and summarization. For example, a system that had extracted the two relations CEO(Steve Jobs, Apple) and CEO(Tim Cook, Apple) wouldn’t be able to answer the question “Who is the CEO of Apple?” It would also need to understand the temporal relationship between the two relations.


%Figure~\ref{fig:motivating-log} presents a log snippet from a hypothetical

%%%%%%%%%%%%%%%%%%%%%%%%%%%%%%%%%%%%%%%%%%%%%%%%%%%%%%%%%%
\subsection{Outline}
%%%%%%%%%%%%%%%%%%%%%%%%%%%%%%%%%%%%%%%%%%%%%%%%%%%%%%%%%%
In section 2 we cover previous work on this problem. Then, in section 3, we go over our representation and approach to parsing. Section 4 reveals our learning. Section 5 shows our results. Finally, section 6 talks about future work.


