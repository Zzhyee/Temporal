%%%%%%%%%%%%%%%%%%%%%%%%%%%%%%%%%%%%%%%%%%%%%%%%%%%%%%%%%%
\subsection*{Abstract}
%%%%%%%%%%%%%%%%%%%%%%%%%%%%%%%%%%%%%%%%%%%%%%%%%%%%%%%%%%
Developers often struggle to understand their systems. In prior work
we created Synoptic, a tool to help developers 
by inferring a concise and accurate system model from execution logs.
However, Synoptic has numerous limitations: (1) it is slow when run on
large input logs, (2) it is non-deterministic and does not always generate
optimal models, (3) it is difficult to extend, and (4) the Synoptic algorithm is complex and
difficult for users to understand.

To remedy the above limitations we developed InvariMint,
which reformulates the core Synoptic process in terms of formal
languages. InvariMint generates a DFA for each log invariant, intersects the 
invariant DFAs, and minimizes the resulting model.
This model inference approach is independent of the size of the log,
deterministic, easy to extend, and uses standard techniques that are easy
to understand. We have also found that the InvariMint approach generalizes
to other algorithms used in model inference, such as the k-Tails algorithm.
We think that InvariMint can act as a common denominator in model
inference, using which researchers can easily compare a wide range of
algorithms and which provides users with a simple process for deriving
customized hybrid algorithms.
